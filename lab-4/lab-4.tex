\documentclass[a4paper,11pt]{article} %опция twoside нужна для разных колонтитулов; если на всех страницах нужен один колонтитул, убери

%%% Работа с русским языком
\usepackage{mathtext} 				% русские буквы в формулах
\usepackage[T2A]{fontenc}			% кодировка
\usepackage[utf8]{inputenc}			% кодировка исходного текста
\usepackage[english,russian]{babel}	% локализация и переносы

%%% Дополнительная работа с математикой
\usepackage{amsmath, amssymb, eucal}

\usepackage{geometry} % Простой способ задавать поля
\geometry{top=20mm}
\geometry{bottom=20mm}
\geometry{left=25mm}
\geometry{right=15mm}

\usepackage[pdftex,unicode, bookmarks, pagebackref]{hyperref} %colorlinks,

%%% Заголовок
\author{Оформление документов}
\title{Лабораторная работа №3}
\date{07 апреля 2022 г.}


\begin{document} % Конец преамбулы, начало текста.
	
	\maketitle % печатает заголовок, имя автора и дату
	
	%%%%%%%%%%%%%%%%%%%%%%%%%%%%%%%%%%%%%%%%%%%%%%%%%%%%%%%%%%%%%%%
	\section{Первое задание}
	\qquad Расписание:
	\begin{itemize}
		\item {
			\texttt{понедельник}
			\begin{itemize}
				\item[\checkmark]{\it 08:15 - 09:35 \\
					\bf Алгебра (лк)}
				\item[\checkmark]{\it 09:45 - 11:05 \\
					\bf Алгебра (пз)}
				\item[\checkmark]{\it 11:15 - 12:35 \\
					\bf Математический анализ (лк)}
			\end{itemize}	
		}
		\item {
			\texttt{вторник}
			\begin{itemize}
				\item[\checkmark]{\it 08:15 - 09:35 \\
					\bf Геометрия (лк)}
				\item[\checkmark]{\it 09:45 - 11:05 \\
					\bf Математический анализ (пз)}
				\item[\checkmark]{\it 11:15 - 12:35 \\
					\bf Методы программирования (пз)}
				\item[\checkmark]{\it 13:00 - 14:20 \\
					\bf Методы программирования (лк)} 
			\end{itemize}	
		}
		\item {
			\texttt{среда}
			\begin{itemize}
				\item[\checkmark]{\it 08:15 - 09:35 \\
					\bf Английский (пз)}
				\item[\checkmark]{\it 09:45 - 11:05 \\
					\bf Физ-ра (пз)}
				\item[\checkmark]{\it 11:15 - 12:35 \\
					\bf Математический анализ (пз)}
			\end{itemize}	
		}
		\item {
			\texttt{четверг}
			\begin{itemize}
				\item[\checkmark]{\it 08:15 - 09:35 \\
					\bf Вычислительная практика (пз)}
				\item[\checkmark]{\it 09:45 - 11:05 \\
					\bf Английский (пз)}
				\item[\checkmark]{\it 11:15 - 12:35 \\
					\bf Математический анализ (лк)}
			\end{itemize}	
		}
		\item {
			\texttt{пятница}
			\begin{itemize}
				\item[\checkmark]{\it 13:00 - 14:20 \\
					\bf Дискретная математика (лк)} 
			\end{itemize}	
		}
		\item {
			\texttt{суббота}
			\begin{itemize}
				\item[\checkmark]{\it 08:15 - 09:35 \\
					\bf Компьютерная математика (пз)}
				\item[\checkmark]{\it 09:45 - 11:05 \\
					\bf Дискретная математика (пз)}
				\item[\checkmark]{\it 11:15 - 12:35 \\
					\bf Геометрия (пз)}
			\end{itemize}	
		}
	\end{itemize}
	
	%%%%%%%%%%%%%%%%%%%%%%%%%%%%%%%%%%%%%%%%%%%%%%%%%%%%%%%%%%%%%%%
	\section{Второе задание}
	\qquad Цитаты Киану Ривза, которые <<взорвали>> социальные сети:
	\begin{quote}
		<<\,\glqq Пустое\grqq\,>> <<\,\glqq сердце\grqq\,>> <<\,\glqq бьётся\grqq\,>> <<\,\glqq ровно…\grqq\,>>
	\end{quote}
	\begin{quote}
		--- Вы не Достоевский, --- сказала гражданка, сбиваемая с толку Коровьевым. \\
		--- Ну, почем знать, почем знать, --- ответил тот. \\
		--- Достоевский умер, — сказала гражданка, но как-то не очень уверенно. \\
		--- Протестую, --- горячо воскликнул Бегемот. --- Достоевский бессмертен!
	\end{quote}
	\begin{quote}
		<<Единственная правота --- доброта>>.
	\end{quote}
	\begin{quote}
		<<Мы уходим, а красота остаётся. Ибо мы направляемся к будущему, а красота есть вечное настоящее>>.
	\end{quote}
	\begin{quote}
		<<Человек – не сумма того, во что он верит или на что он уповает. Не сумма своих убеждений. Человек --- сумма своих поступков>>.
	\end{quote}
	\begin{quote}
		\begin{center}
			Смотри без суеты вперед. \\
			Назад без ужаса смотри. \\
			Будь прям и горд, \\
			Раздроблен изнутри, на ощупь тверд.
		\end{center}
	\end{quote}
	\begin{quote}
		\begin{center}
			Не уходи смиренно, в сумрак вечной тьмы, \\
			Пусть тлеет бесконечность в яростном закате. \\
			Пылает гнев на то, как гаснет смертный мир, \\
			Пусть мудрецы твердят, что прав лишь тьмы покой. \\
			И не разжечь уж тлеющий костёр. \\
			Не уходи смиренно в сумрак вечной тьмы, \\
			Пылает гнев на то, как гаснет смертный мир \\
			Не гасни, уходя во мрак ночной. \\
			Пусть вспыхнет старость заревом заката. \\
			Встань против тьмы, сдавившей свет земной. \\
			Мудрец твердит: ночь — праведный покой, \\
			Не став при жизни молнией крылатой. \\
			Не гасни, уходя во мрак ночной. \\
			Глупец, побитый штормовой волной, \\
			Как в тихой бухте — рад, что в смерть упрятан... \\
			Встань против тьмы, сдавившей свет земной. \\
			Подлец, желавший солнце скрыть стеной, \\
			Скулит, когда приходит ночь расплаты. \\
			Не гасни, уходя во мрак ночной. \\
			Слепец прозреет в миг последний свой: \\
			Ведь были звёзды-радуги когда-то... \\
			Встань против тьмы, сдавившей свет земной. \\
			Отец, ты — перед чёрной крутизной. \\
			От слёз всё в мире солоно и свято. \\
			Не гасни, уходя во мрак ночной. \\
			Встань против тьмы, сдавившей свет земной.
		\end{center}
	\end{quote}
	\begin{quote}
		<<Молодежь может не знать, что такое старость. Но я ненавижу взрослых людей, которые забыли, что значит быть молодым>>.
	\end{quote}
	\begin{quote}
		<<Не жалею, не зову, не плачу>>.
	\end{quote}
	\begin{quote}
		<<Я могу принять поражение, но я не могу принять отсутствие попыток>>.
	\end{quote}
	
	%%%%%%%%%%%%%%%%%%%%%%%%%%%%%%%%%%%%%%%%%%%%%%%%%%%%%%%%%%%%%%%
	\section{Третье задание}
	
	%%%%%%%%%%%%%%%%%%%%%%%%%%%%%%%%%%%%%%%%%%%%%%%%%%%%%%%%%%%%%%%
	\section{Четвертое задание}
	
	%%%%%%%%%%%%%%%%%%%%%%%%%%%%%%%%%%%%%%%%%%%%%%%%%%%%%%%%%%%%%%%
	\section{Пятое задание}
	
\end{document} % Конец текста.
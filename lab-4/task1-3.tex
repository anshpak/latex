\documentclass[a4paper,12pt]{article} %опция twoside нужна для разных колонтитулов; если на всех страницах нужен один колонтитул, убери

%%% Работа с русским языком
\usepackage{cmap}					% поиск в PDF
\usepackage{mathtext} 				% русские буквы в формулах
\usepackage[T2A]{fontenc}			% кодировка
\usepackage[utf8]{inputenc}			% кодировка исходного текста
\usepackage[english,russian]{babel}	% локализация и переносы

%%% Дополнительная работа с математикой
\usepackage{amsmath,amsfonts,amssymb,amsthm,mathtools} % AMS
\usepackage{icomma} % "Умная" запятая: $0,2$ --- число, $0, 2$ --- перечисление

\usepackage{geometry} % Простой способ задавать поля
\geometry{top=20mm}
\geometry{bottom=20mm}
\geometry{left=25mm}
\geometry{right=15mm}

\begin{document} % Конец преамбулы, начало текста.

\section{Перечни}
Создайте вложенный перечень глубины 2 (список в списке) на любую тему с изменением шрифтов некоторых слов.
Что-то в духе <<Дни недели>>:
\begin{itemize}
	\item {\bf понедельник}
	\item {\it вторник}
	\item \textit{\textup{(} среда \textup{)}}
	\item \texttt{четверг}
	\item \textit{(пятница) }
	\item суббота
	\item воскресенье
\end{itemize}

\section{Цитаты}
Процитируйте 10 предложений из русской классической литературы (пусть цитата содержит кавычки внутри кавычек).

Сейчас будет цитата:
\begin{quote}
	А вот, собственно, и она. Как легко видеть, имеет она свои отступы. Текст текст текст текст текст текст текст текст текст текст текст текст текст текст текст текст текст текст текст текст текст текст текст текст текст текст текст текст текст текст текст текст текст текст текст текст текст текст текст текст текст текст текст текст текст текст текст текст текст текст текст текст текст текст текст текст текст текст текст текст текст текст текст текст текст текст текст текст текст текст текст текст текст текст текст текст текст текст текст текст текст текст текст текст текст
\end{quote}
И дальше текст продолжается

\section{Оформление документа}
Оформите текст по аналогии с приведенным в task3:
\begin{itemize}
	\item Класс article, шрифт 11pt (до конца работы включите опцию draft);
	\item поля: сверху 2cm, снизу 2cm, слева 2,5cm, справа 1,5cm;
	\item разные колонтитулы для четных и нечетных страниц;
	\item для первой страницы колонтитулов нет;
	\item сноска на первой странице;
	\item отсутствие переполненных строк (следите и за переносами слов);
	\item аннотация;
	\item список литературы.
\end{itemize}
Следите за правильностью переносов и шрифтами! Все ссылки должны быть кликабельными!

\end{document} % Конец текста.
\documentclass[a4paper,12pt]{article} % добавить leqno в [] для нумерации слева

%%% Работа с русским языком
\usepackage{cmap}					% поиск в PDF
\usepackage{mathtext} 				% русские буквы в формулах
\usepackage[T2A]{fontenc}			% кодировка
\usepackage[utf8]{inputenc}			% кодировка исходного текста
\usepackage[english,russian]{babel}	% локализация и переносы

%%% Дополнительная работа с математикой
\usepackage{amsmath,amsfonts,amssymb,amsthm,mathtools} % AMS
\usepackage{icomma} % "Умная" запятая: $0,2$ --- число, $0, 2$ --- перечисление

%% Номера формул
%\mathtoolsset{showonlyrefs=true} % Показывать номера только у тех формул, на которые есть \eqref{} в тексте.

%%% Работа с таблицами
\usepackage{array,tabularx,tabulary,booktabs} % Дополнительная работа с таблицами
\usepackage{longtable}  % Длинные таблицы
\usepackage{multirow} % Слияние строк в таблице
\usepackage{wrapfig} % Обтекание рисунков и таблиц текстом

%% Шрифты
\usepackage{euscript}	 % Шрифт Евклид
\usepackage{mathrsfs} % Красивый матшрифт

%% Свои команды
\DeclareMathOperator{\sgn}{\mathop{sgn}}

%%% Заголовок
\author{Важные детали}
\title{Лабораторная работа №3}
\date{24 марта 2022 г.}


\begin{document} % Конец преамбулы, начало текста.
	
	\maketitle % печатает заголовок, имя автора и дату
	
	%%%%%%%%%%%%%%%%%%%%%%%%%%%%%%%%%%%%%%%%%%%%%%%%%%%%%%%%%%%%%%%
	\section{Первое задание}	
	$\quad$ Пример 18. Вычислить интеграл
	\begin{equation} \label{eleven}
		\displaystyle \int \limits_{-1}^{2} \dfrac{1 + x^2}{1 + x^4} \; dx. \tag{11}
	\end{equation}

	$\blacktriangle$
	Сделав при 
	$x \neq 0$
	замену переменного 
	$t = x - 1 / x$
	в соответствующем неопределенном интеграле, получим
	$$
		\displaystyle \int \dfrac{1 + x^2}{1 + x^4} \: dx =
		\int \dfrac{d \left(x - 1 / x \right)}{2 + \left(x - 1 / x \right)^2} = 
		\dfrac{1}{\sqrt{2}} \: \arctg \: \dfrac{x^2 - 1}{x \sqrt{2}} + C
	$$
	и, следовательно,
	\begin{equation} \label{twelve}
		\left( \dfrac{1}{\sqrt{2}} \: \arctg \: \dfrac{x^2 - 1}{x \sqrt{2}} \right)' = 
		\dfrac{1 + x^2}{1 + x^4}, \; \; \;
		x \neq 0. \tag{12}
	\end{equation}

	Поскольку
	$$
		\lim \limits_{x\to +0} \dfrac{1}{\sqrt{2}} \: \arctg \: \dfrac{x^2 - 1}{x \sqrt{2}} = 
		- \dfrac{\pi}{2 \sqrt{2}} \; \; \;
		\lim \limits_{x\to -0} \dfrac{1}{\sqrt{2}} \: \arctg \: \dfrac{x^2 - 1}{x \sqrt{2}} = 
		\dfrac{\pi}{2 \sqrt{2}},
	$$
	то функция
	\[
		F(x)=
		\begin{cases} \label{thirteen}
			\dfrac{1}{\sqrt{2}} \: \arctg \: \dfrac{x^2 - 1}{x \sqrt{2}} -
			\dfrac{\pi}{2 \sqrt{2}}, &\text{если } \: \: \: x > 0, 
			\\
			\hfil 0, &\text{если } \: \: \: x = 0.
			\\
			\dfrac{1}{\sqrt{2}} \: \arctg \: \dfrac{x^2 - 1}{x \sqrt{2}} + 
			\dfrac{\pi}{2 \sqrt{2}}, &\text{если } \: \: \: x < 0.
			\tag{13}
		\end{cases}
	\]
	будет непрерывной на всей числовой оси, а так как согласно $\eqref{twelve}$
	\begin{equation} \label{fourteen}
		F'(x) = \dfrac{1 + x^2}{1 + x^4}, \quad x \neq 0,
		\tag{14}
	\end{equation}
	то в силу непрерывности функции
	$\left(1 + x^2\right) \left(1 + x^4\right)$
	равенство $\eqref{fourteen}$ верно и при $x = 0$. Таким образом, функция $\eqref{thirteen}$ является первообразной для подынтегральной функции интеграла $\eqref{eleven}$. Поэтому
	$$
		\displaystyle \int \limits_{-1}^{2} \dfrac{1 + x^2}{1 + x^4} \; dx =
		F(2) - F(-1) = 
		\dfrac{1}{\sqrt{2}} \left( \arctg \dfrac{3 \sqrt{2}}{4} + \pi \right). \; \blacktriangle
	$$
	
	%%%%%%%%%%%%%%%%%%%%%%%%%%%%%%%%%%%%%%%%%%%%%%%%%%%%%%%%%%%%%%%
	\section{Второе задание}
	$$
		\omega_i(f) = \sup \limits_{x' \in \left[a ; b\right] \atop x'' \in \left[a ; b\right]} |f(x') -f(x'')| \leqslant \varepsilon, \quad
		i=1,\, 2, \, ... , \, k_\tau,
	$$
	а поэтому
	$$
		\sum \limits_{i=1}^{k_\tau} \omega_i(f) \Delta x_i \leqslant \varepsilon 
		\sum \limits_{i=1}^{k_\tau} \Delta x_i = 
		\varepsilon \left(b - a\right).
	$$
	Отсюда
	$$
		\lim \limits_{| \tau | \to 0} \sum \limits_{i=1}^{k_\tau} \omega_i(f) \Delta x_i = 0,
	$$
	
	%%%%%%%%%%%%%%%%%%%%%%%%%%%%%%%%%%%%%%%%%%%%%%%%%%%%%%%%%%%%%%%
	\section{Третье задание}
	$\quad$ Пример 13. Найти предел последовательности 
	$$
		s_n = \dfrac{1^{\alpha} + 2^{\alpha} + \dots + n^{\alpha}}{n^{\alpha + 1}}, \quad \alpha > 1.
	$$
	$\quad \blacktriangle$ Поскольку сумма $s_n \! = \dfrac{1}{n} \displaystyle \sum \limits_{k-1}^n \left( \dfrac{k}{n} \right)^ \alpha$ является интегральной суммой функции
	$f(x) = x^{\alpha}$ на отрезке $\left[0; 1\right]$, то
	$$
		\lim \limits_{n \to \infty} s_n = 
		\int \limits_0^1 x^{\alpha} \; dx = 
		\left. \dfrac{x^{\alpha + 1}}{\alpha + 1} \right|_0^1 =
		\dfrac{1}{\alpha + 1}. \; \blacktriangle
	$$
	
	%%%%%%%%%%%%%%%%%%%%%%%%%%%%%%%%%%%%%%%%%%%%%%%%%%%%%%%%%%%%%%%
	\section{Четвертое задание}
	$\quad$ 3) Используя равенство
	$
		\dfrac{1}{k \left(k + m\right)} = \dfrac{1}{m}
		\left( \dfrac{1}{k} - \dfrac{1}{k + m} \right) \! ,
	$
	получаем
	$$
		S_n = \sum \limits_{k = 1}^n \dfrac{1}{k \left(k + m\right)} = 
		\dfrac{1}{m} \left( \sum \limits_{k=1}^n \dfrac{1}{k} - \sum \limits_{k=1}^n \dfrac{1}{k + m} \right) = 
		\dfrac{1}{m} \sum \limits_{k = 1}^n \dfrac{1}{k} -
		\dfrac{1}{m} \sum \limits_{k = n + 1}^{n + m} \dfrac{1}{k},
	$$
	откуда следует, что
	$$
		\lim \limits_{n \to \infty} S_n = \dfrac{1}{m} \sum \limits_{k=1}^m \dfrac{1}{k},
	$$
	т. е.
	$$
		\sum \limits_{n = 1}^{\infty} \dfrac{1}{n \left(n + m\right)} =
		\dfrac{1}{m} \left(1 + \dfrac{1}{2} + \dots + \dfrac{1}{m} \right) \! . \; \blacktriangle
	$$
	
	%%%%%%%%%%%%%%%%%%%%%%%%%%%%%%%%%%%%%%%%%%%%%%%%%%%%%%%%%%%%%%%
	\section{Пятое задание}
	$\quad$ Определение предела:
	$$
		\lim \limits_{x \to a} f(x) = A \stackrel{def}{\Longleftrightarrow}
		\forall \varepsilon > 0 \exists \delta_{\varepsilon} : 
		0 < |x - a| < \delta_{\varepsilon} \Longrightarrow 
		| f(x) - A | < \varepsilon.
	$$
	$$
		M \stackrel{def}{=} \left\{ f \mid f : \mathbb{N} \to \mathbb{N} : n \longmapsto \underbrace{1 + \dots + }_{n \text{ единиц}} \right\}
	$$

\end{document} % Конец текста.
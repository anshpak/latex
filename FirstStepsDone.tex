\documentclass[a4paper,12pt]{article} % добавить leqno в [] для нумерации слева

%%% Работа с русским языком
\usepackage{cmap}					% поиск в PDF
\usepackage{mathtext} 				% русские буквы в формулах
\usepackage[T2A]{fontenc}			% кодировка
\usepackage[utf8]{inputenc}			% кодировка исходного текста
\usepackage[english,russian]{babel}	% локализация и переносы

%%% Дополнительная работа с математикой
\usepackage{amsmath,amsfonts,amssymb,amsthm,mathtools} % AMS
\usepackage{icomma} % "Умная" запятая: $0,2$ --- число, $0, 2$ --- перечисление

%% Номера формул
%\mathtoolsset{showonlyrefs=true} % Показывать номера только у тех формул, на которые есть \eqref{} в тексте.

%% Шрифты
\usepackage{euscript}	 % Шрифт Евклид
\usepackage{mathrsfs} % Красивый матшрифт

%% Свои команды
\DeclareMathOperator{\sgn}{\mathop{sgn}}

%% Перенос знаков в формулах (по Львовскому)
\newcommand*{\hm}[1]{#1\nobreak\discretionary{}
	{\hbox{$\mathsurround=0pt #1$}}{}}

%%% Заголовок
\author{\LaTeX{} в Вышке}
\title{1.2 Математика в \LaTeX}
\date{\today}


\begin{document} % Конец преамбулы, начало текста.

\maketitle % печатает заголовок, имя автора и дату


\section{Первые шаги}
\subsection{Текст формулы. Абзац}

Привет,                                          Мир!

Абзац.
Всё ещё первый абзац.
И это тоже всё ещё первый абзац.
				
Второй абзац.







Третий абзац. 
$ 2 + 2 = 4 $. 
И продолжим текст этого абзаца. Текст Текст Текст Текст Текст Текст Текст Текст Текст Текст Текст Текст екст Текст Текст Текст Текст Текст Текстекст Текст Текст Текст Текст Текст Текстекст Текст Текст Текст Текст Текст Текстекст Текст Текст Текст Текст Текст Текстекст Текст Текст Текст Текст Текст Текстекст Текст Текст Текст Текст Текст Текстекст Текст Текст Текст Текст Текст Текстекст Текст Текст Текст Текст Текст Текстекст Текст Текст Текст Текст Текст Текстекст Текст Текст Текст Текст Теывавапварпапрапрапрвпаывакст Текстекст Текст Текст Текст Текст Текст Текстекст Текст Текст Текст Текст Текст Текст 

$1+2+3+4+5=15$

$ 1 + 2 + 3 + 4 + 5 = 15 $

\subsection{Выключная  и строчная формулы}

Я люблю математику. 

$ 2 + 2 = 4 $

Выключная формула

\[ 2+2=  4 \]

или так

$$2+2=4$$

а можно и с номером

\begin{equation}\label{eq:firstEquationInMyLife}
	2+2=4
\end{equation}
если номер не нужен его легко убрать
\begin{equation*}
	2+2=4
\end{equation*}

\subsection{Ссылки}


Поверьте! Формула \eqref{eq:firstEquationInMyLife} на стр. \pageref{eq:firstEquationInMyLife} не единственная формула, которую я знаю.

\ref{eq:firstEquationInMyLife} --- просто ссылка

\section{Набор формул}

\subsection{Дроби}
ыРПлоталпрдлвьжапвлапыжбваьпылптойлдуьацуа
$ \frac{6}{3}=2 $, % тут размер Latex сам подгоняет размеры сам
влоапвиргшптопшщуьлдкпдукжбпуждьлпкрешьлдкртьабпьидвлтшпащуьклд
$ \dfrac{6}{3}=2$. % а тут мы требуем не менять размеры дроби

% $$ тут находится выключная формула (без номера)  $$

$$
\frac{6}{3}=2.
$$

$$
\frac{5+\frac{2}{1}}{3}=2.
$$

$$
\frac{5+\dfrac{2}{1}}{3}=2.
$$

$$
\frac{1}{2}=0,5
$$

$0,5$  $(0, 5)$ % умная запятая

\subsection{Скобки}

$$
(2+3)\times2=10
$$

$$
(1+\frac{1}{2})\cdot 2=3
$$

$$
\left\{10+1=11 + \dfrac{1}{2} \right\}
$$

$$
\left(1+\frac{1}{2}\right) \cdot 2=3
$$

$$
\left[1+\frac{1}{2}\right] \cdot 2=3
$$

$$
{2+3}\cdot2=10
$$

$$
\{2+\frac{3}{2}\}\cdot2=5
$$

$$
\left\{2+\frac{3}{2}\right\}\cdot2=5
$$

$$
1+2+\ldots+n=\frac{(1+n)n}{2}
$$

\subsection{Степени  и индексы}

$2^2=4$ $2^{x+y}$

$x_1$, $x_12$, $x_{12}$

Длина отрезка $AB$ с концами $A\left(x_1, y_1\right) $ и $B\left(x_2, y_2\right) $ равна

$$
|AB|=\left( \left( x_2-x_1\right)^2+\left(y_2-y_1 \right)^2  \right) ^{\frac{1}{2}}
$$

\subsection{Символы}

$A \cap B, A\cup B,  A\setminus B, n\to\infty,  \Rightarrow, \Longleftarrow, x\ne y, \le, \ge, a\in A, A\subset B, \forall, \exists,$

 $\int$,  $\int_{a}^{b}$, $\lim$, $\lim_{n\to\infty}$, $\lim\limits_{n\to\infty}$
 
 $$
 \lim\limits_{n\to\infty}a_n=a \Longleftrightarrow \forall \varepsilon>0 \exists N_{\varepsilon}\in\mathbb{N} \forall n\ge N_{\varepsilon} \left|a_n-a\right|<\varepsilon.
 $$

\subsection{Буквы других алфавитов}

$\pi, \alpha, \gamma, \omega, \phi, \sigma$, $\aleph$

$\Gamma, \Phi, \Sigma$

$\epsilon, \phi$

$\varepsilon, \varphi$

\subsection{Функции}

$ sin x=1 $, % вот так писать плохо
$ \sin (xfdgdfghf)=1$, % а вот так хорошо
$\sin(xy)=1$, $\sin\pi=0$, $1+\tg^2\alpha=\frac{1}{\cos^2\alpha}$

 $\ln2$, $\log_2x$
 
$\sin (328x + 15) = 0,1$


$\sgn x=0$, $\sqrt{x+y}$ $\sqrt[15]{x+y}$

\subsection{Диакритические знаки}

$\bar a, \tilde{a}$

$\bar xy,\bar{xy}, \overline{xy}, \overbrace{xy},  \underline{xy}$

\subsection{Одно над другим}

1) Аналогично дроби, но без черты: $ij\atop k$ $n \choose k$

2) Нижняя часть в строке, верхняя выше: $\stackrel{f}{\longrightarrow}, \stackrel{def}{=}$

3) Фигурная скобка с текстом $\underbrace{a_1+\ldots+a_n}_{n \text{ слагаемых}}$ $\overbrace{a_1+\ldots+a_n}^{n \text{ слагаемых}}$



\subsection{Многострочные формулы}

$$
2\times 2=4;
3\times 3=9
$$

$$
2\times 2=4;
$$
$$
3\times 3=9
$$

\begin{multline}
	1+2+3+4+5+6+7+\ldots+\\+50+51+52+53+54+55+56+\ldots+\\+95+96+97+98+99+100=5050;
\end{multline}

\begin{align}
	2\times 2&=4  & 4\times 4=16 \\
	3\times 3&=9 & 5\times 5=25 \\
	52021 =520&21 & 6\times 6 =36
\end{align}

$$
\begin{aligned}
	2\times 2&=4  & 4\times 4=16\\
	3\times 3&=9 & 5\times 5=25\\
	52021&=52021 & 6\times 6 =36
\end{aligned}
$$

\subsection{Системы уравнений}

$$
\left\{
\begin{aligned}
	2\times 2&=4\\
	3\times 3&=9\\
	52021&=52021
\end{aligned}
\right\}
$$

$$
\sgn x= \begin{cases} %выравнивания нет
	1, \text{если } x>0;\\
	0, \text{если } x=0;\\
	-1, \text{если } x<0.
\end{cases}
$$

$$
\sgn x= \begin{cases} %а тут есть
	1, &\text{если } x>0;\\
	0, &\text{если } x=0;\\
	-1, &\text{если } x<0.
\end{cases}
$$

\subsection{Матрицы}

$$
\begin{pmatrix}
	a_{11} & a_{12} & a_{13}\\
	a_{21} & a_{22} & a_{23}
\end{pmatrix}
$$

$$
\begin{vmatrix}
	a_{11} & a_{12} & a_{13}\\
	a_{21} & a_{22} & a_{23}
\end{vmatrix}
$$

$$
\begin{bmatrix}
	a_{11} & a_{12} & a_{13}\\
	a_{21} & a_{22} & a_{23}
\end{bmatrix}
$$

$$
\begin{array}{lrcc}
	a_{11} & a_{12} & a_{13} & a_{14}\\
	a_{21} & a_{22} & a_{23} & a_{24}
\end{array}
$$

$$
\left\{
\begin{array}{rcl}
	2\times 2 & = & 4\\
	3\times 3 & = & 9\\
	152021 & = & 152021
\end{array}
\right.
$$


\end{document} % Конец текста.

% $Header: /cvsroot/latex-beamer/latex-beamer/solutions/conference-talks/conference-ornate-20min.en.tex,v 1.7 2007/01/28 20:48:23 tantau Exp $

%%%%%%%%%%%%%%%%%%%%%%%%%%%%%%%%%%%%%%%%%%%%%%%%%%%%%%%%%%%%%%%%%%%
%%%%%%%%%%%%%%%%%%%%%%%%%%%%%%%%%%%%%%%%%%%%%%%%%%%%%%%%%%%%%%%%%%%
%%%%%%%%%%%%%%%%%%%%%%%%%%%%%%%%%%%%%%%%%%%%%%%%%%%%%%%%%%%%%%%%%%%

\documentclass{beamer}
\usepackage{amsmath, amssymb, eucal}
\usepackage[english]{babel}
\usepackage[cp1251]{inputenc}
\usepackage[T2A]{fontenc}
\usepackage{esint}
% Or whatever. Note that the encoding and the font should match. If T1
% does not look nice, try deleting the line with the fontenc.
% This file is a solution template for:

% - Talk at a conference/colloquium.
% - Talk length is about 20min.
% - Style is ornate.

%%%%%%%%%%%%%%%%%%%%%%%%%%%%%%%%%%%%%%%%%%%%%%%%%%%%%%%%%%%%%%%%%%%
%%%%%%%%%%%%%%%%%%%%%%%%%%%%%%%%%%%%%%%%%%%%%%%%%%%%%%%%%%%%%%%%%%%
%%%%%%%%%%%%%%%%%%%%%%%%%%%%%%%%%%%%%%%%%%%%%%%%%%%%%%%%%%%%%%%%%%%

% Copyright 2004 by Till Tantau <tantau@users.sourceforge.net>.
%
% In principle, this file can be redistributed and/or modified under
% the terms of the GNU Public License, version 2.
%
% However, this file is supposed to be a template to be modified
% for your own needs. For this reason, if you use this file as a
% template and not specifically distribute it as part of a another
% package/program, I grant the extra permission to freely copy and
% modify this file as you see fit and even to delete this copyright
% notice.



%\mode<presentation>
{
\usetheme{CambridgeUS}
  % or ...
\setbeamertemplate{theorems}[numbered]
\setbeamercovered{transparent}
  % or whatever (possibly just delete it)
}

%��������� ����� ��� �������������� ������
\usefonttheme[onlymath]{serif}

%%%%%%%%%%%%%%%%%%%%%%%%%%%%%%%%%%%%%%%%%%%%%%%%%%%%%%%%%%%%%%%%%%%
%%%%%%%%%%%%%%%%%%%%%%%%%%%%%%%%%%%%%%%%%%%%%%%%%%%%%%%%%%%%%%%%%%%
%%%%%%%%%%%%%%%%%%%%%%%%%%%%%%%%%%%%%%%%%%%%%%%%%%%%%%%%%%%%%%%%%%%

\makeatletter
\renewcommand{\@begintheorem}[2]{\trivlist \item[\hskip\labelsep{\bfseries #1\ #2.}]\itshape}
\makeatother

%\newtheorem{theorem}{\hskip\parindent Theorem}
%\newtheorem{lemma}{\hskip\parindent Lemma}
%\newtheorem{corollary}{\hskip\parindent Corollary}

%%%%%%%%%%%%%%%%%%%%%%%%%%%%%%%%%%%%%%%%%%%%%%%%%%%%%%%%%%%%%%%%%%%
%%%%%%%%%%%%%%%%%%%%%%%%%%%%%%%%%%%%%%%%%%%%%%%%%%%%%%%%%%%%%%%%%%%
%%%%%%%%%%%%%%%%%%%%%%%%%%%%%%%%%%%%%%%%%%%%%%%%%%%%%%%%%%%%%%%%%%%

\include{skCommands}

%%%%%%%%%%%%%%%%%%%%%%%%%%%%%%%%%%%%%%%%%%%%%%%%%%%%%%%%%%%%%%%%%%%


\title[Approximations of identity] % (optional, use only with long paper titles)
{Approximations of identity on metric measure spaces}

\author{V.G. Krotov} % (optional, use only with lots of authors)
% - Give the names in the same order as the appear in the paper.
% - Use the \inst{?} command only if the authors have different
%   affiliation.
%\institute[BSU]{Belarusian State University} % (optional, but mostly needed)

\date{Jena, 13.12.2019} % (optional, should be abbreviation of conference name)
% - Either use conference name or its abbreviation.
% - Not really informative to the audience, more for people (including
%   yourself) who are reading the slides online

%\subject{Theoretical Computer Science}
% This is only inserted into the PDF information catalog. Can be left
% out.

% If you have a file called "university-logo-filename.xxx", where xxx
% is a graphic format that can be processed by latex or pdflatex,
% resp., then you can add a logo as follows:

% \pgfdeclareimage[height=0.5cm]{university-logo}{university-logo-filename}
% \logo{\pgfuseimage{university-logo}}



% Delete this, if you do not want the table of contents to pop up at
% the beginning of each subsection:

%%%%%%%%%%%%%%%%%%%%%%%%%%%%%%%%%%%%%%%%%%%%%%%%%%%%%%%%%%%%%%%%%%%
%%%%%%%%%%%%%%%%%%%%%%%%%%%%%%%%%%%%%%%%%%%%%%%%%%%%%%%%%%%%%%%%%%%
%%%%%%%%%%%%%%%%%%%%%%%%%%%%%%%%%%%%%%%%%%%%%%%%%%%%%%%%%%%%%%%%%%%

% If you wish to uncover everything in a step-wise fashion, uncomment
% the following command:

%\beamerdefaultoverlayspecification{<+->}

%%%%%%%%%%%%%%%%%%%%%%%%%%%%%%%%%%%%%%%%%%%%%%%%%%%%%%%%%%%%%%%%%%%
%%%%%%%%%%%%%%%%%%%%%%%%%%%%%%%%%%%%%%%%%%%%%%%%%%%%%%%%%%%%%%%%%%%
%%%%%%%%%%%%%%%%%%%%%%%%%%%%%%%%%%%%%%%%%%%%%%%%%%%%%%%%%%%%%%%%%%%

\begin{document}
	
%%%%%%%%%%%%%%%%%%%%%%%%%%%%%%%%%%%%%%%%%%%%%%%%%%%%%%%%%%%%%%%%%%%
%%%%%%%%%%%%%%%%%%%%%%%%%%%%%%%%%%%%%%%%%%%%%%%%%%%%%%%%%%%%%%%%%%%
%%%%%%%%%%%%%%%%%%%%%%%%%%%%%%%%%%%%%%%%%%%%%%%%%%%%%%%%%%%%%%%%%%%

\begin{frame}
  \titlepage
\end{frame}

%%%%%%%%%%%%%%%%%%%%%%%%%%%%%%%%%%%%%%%%%%%%%%%%%%%%%%%%%%%%%%%%%%%
%%%%%%%%%%%%%%%%%%%%%%%%%%%%%%%%%%%%%%%%%%%%%%%%%%%%%%%%%%%%%%%%%%%
%%%%%%%%%%%%%%%%%%%%%%%%%%%%%%%%%%%%%%%%%%%%%%%%%%%%%%%%%%%%%%%%%%%


% Structuring a talk is a difficult task and the following structure
% may not be suitable. Here are some rules that apply for this
% solution:

% - Exactly two or three sections (other than the summary).
% - At *most* three subsections per section.
% - Talk about 30s to 2min per frame. So there should be between about
%   15 and 30 frames, all told.

% - A conference audience is likely to know very little of what you
%   are going to talk about. So *simplify*!
% - In a 20min talk, getting the main ideas across is hard
%   enough. Leave out details, even if it means being less precise than
%   you think necessary.
% - If you omit details that are vital to the proof/implementation,
%   just say so once. Everybody will be happy with that.

%%%%%%%%%%%%%%%%%%%%%%%%%%%%%%%%%%%%%%%%%%%%%%%%%%%%%%%%%%%%%%%%%%%
%%%%%%%%%%%%%%%%%%%%%%%%%%%%%%%%%%%%%%%%%%%%%%%%%%%%%%%%%%%%%%%%%%%
%%%%%%%%%%%%%%%%%%%%%%%%%%%%%%%%%%%%%%%%%%%%%%%%%%%%%%%%%%%%%%%%%%%

\section{Approximations of identity on $\bbR^d$}

%%%%%%%%%%%%%%%%%%%%%%%%%%%%%%%%%%%%%%%%%%%%%%%%%%%%%%%%%%%%%%%%%%%
%%%%%%%%%%%%%%%%%%%%%%%%%%%%%%%%%%%%%%%%%%%%%%%%%%%%%%%%%%%%%%%%%%%
%%%%%%%%%%%%%%%%%%%%%%%%%%%%%%%%%%%%%%%%%%%%%%%%%%%%%%%%%%%%%%%%%%%

\begin{frame}[<+->]\frametitle{}

Consider the function $\varphi\in L^1(\rd)$ and its dilatations
\begin{equation}\label{eqDilatationsOfFunction}
\varphi_t(x)=t^{-d}\varphi\brac{\frac{x}{t}},\quad t>0.
\end{equation}

\pause

The convolutions of the form
\begin{equation}\label{eqApproximationsOfIdentity}
(f\ast \varphi_t) (x)=\int\limits_{\rd }\varphi_t(x-y)f(y)\,dy
\end{equation}
usually called \textbf{approximations of identity}.

\pause


\end{frame}

%%%%%%%%%%%%%%%%%%%%%%%%%%%%%%%%%%%%%%%%%%%%%%%%%%%%%%%%%%%%%%%%%%%

\begin{frame}[<+->]\frametitle{}
	
If the normalization condition is fulfilled
\begin{equation}\label{eqIntOfPhiEqualTo1}
\int\limits_{\rd }\varphi(y)\,dy=1,
\end{equation}
then for any function $f\in C_0(\rd )$ the convolution $f\ast \varphi_t $ converges $t\to+0$ to $f$ uniformly on $\rd$.

\pause

Moreover
\begin{equation*}%\label{eq}
\lim_{t\to+0}\norm{f-f\ast \varphi_t }{L^p(\rd )}=0.
\end{equation*}

\end{frame}

%%%%%%%%%%%%%%%%%%%%%%%%%%%%%%%%%%%%%%%%%%%%%%%%%%%%%%%%%%%%%%%%%%%

\begin{frame}[<+->]\frametitle{}
	
In the study of convergence almost everywhere, an important role is played by the condition of summability of the radial majorant
\begin{equation}\label{eqRadialMajorantIsSummable}
\no{\varphi}\in L^1(\rd ),\quad \no{\varphi}(x)=\sup\{|\varphi(y)|:|y|\ge |x|\}.
\end{equation}

\pause

\begin{figure}[ht]
	\begin{center}	
		\includegraphics{picture1.pdf}
	\end{center}
	%\caption{}
\end{figure}  
	
\end{frame}

%%%%%%%%%%%%%%%%%%%%%%%%%%%%%%%%%%%%%%%%%%%%%%%%%%%%%%%%%%%%%%%%%%%

\begin{frame}[<+->]\frametitle{}
	
\begin{theorem}\label{thConvergenceOfAIAlmostEverywhere}
	%\index[words]{�������!}
	Let the function $\varphi$ �� $\rd $ satisfy the conditions (\ref{eqIntOfPhiEqualTo1}) and (\ref{eqRadialMajorantIsSummable}). Then for any function $f\in L^p(\rd )$, $1\le p<\infty$,
	\begin{equation}\label{eqConvergenceOfAI}
	\lim_{t\to+0}(f\ast \varphi_t )(x)=f(x)
	\end{equation}
	for almost all $x\in\rd $.
\end{theorem}

\pause

In the theorem \ref{thConvergenceOfAIAlmostEverywhere} the <<radial>> convergence can be replaced by <<nontangential>>, that is, we can replace  $\lim_{t\to+0}$ by
\begin{equation*}%\label{eq}
D(x)-\lim,\quad\text{where}\quad D(x)=\{(y,t):|x-y|<at\}
\end{equation*}
for any $a>0$.	
	
\end{frame}


%%%%%%%%%%%%%%%%%%%%%%%%%%%%%%%%%%%%%%%%%%%%%%%%%%%%%%%%%%%%%%%%%%%
%%%%%%%%%%%%%%%%%%%%%%%%%%%%%%%%%%%%%%%%%%%%%%%%%%%%%%%%%%%%%%%%%%%
%%%%%%%%%%%%%%%%%%%%%%%%%%%%%%%%%%%%%%%%%%%%%%%%%%%%%%%%%%%%%%%%%%%

\section{Examples}

%%%%%%%%%%%%%%%%%%%%%%%%%%%%%%%%%%%%%%%%%%%%%%%%%%%%%%%%%%%%%%%%%%%
%%%%%%%%%%%%%%%%%%%%%%%%%%%%%%%%%%%%%%%%%%%%%%%%%%%%%%%%%%%%%%%%%%%
%%%%%%%%%%%%%%%%%%%%%%%%%%%%%%%%%%%%%%%%%%%%%%%%%%%%%%%%%%%%%%%%%%%

\begin{frame}[<+->]\frametitle{}
	
The Steklov means
\begin{equation*}%\label{eq}
\varphi(x)=c_d\chi_{B(0,1)}(x),\quad c_d=\frac{1}{\mes{B(0,1)}},
\end{equation*}	

\end{frame}


%%%%%%%%%%%%%%%%%%%%%%%%%%%%%%%%%%%%%%%%%%%%%%%%%%%%%%%%%%%%%%%%%%%

\begin{frame}[<+->]\frametitle{}

The Poisson kernel
\begin{equation*}%\label{eq}
P(x)=c_d(1+|x|^2)^{-\frac{d+1}{2}},\quad c_d=\pi^{\frac{d+1}{2}}\Gamma\brac{\frac{d+1}{2}}.
\end{equation*}
he constant $c_d$ chosen so that the normalization condition (\ref{eqIntOfPhiEqualTo1}) is satisfied.

\pause

The dilations
\begin{equation}\label{eqRepresentationOfPoissonKernel}
P_t(x)=t^{-d}P\brac{\frac{x}{t}}=c_d\frac{t}{(t^2+|x|^2)^{\frac{d+1}{2}}}
\end{equation}
of function $P$ are called \textbf{the Poisson kernel} of upper hal fspace
\begin{equation*}%\label{eq}
\bbR^{d+1}_{+}=\fbrac{(x,t): x\in\rd ,\quad t>0}. 
\end{equation*}
They are related to the Laplace equation $\Delta u=0$.

\end{frame}

%%%%%%%%%%%%%%%%%%%%%%%%%%%%%%%%%%%%%%%%%%%%%%%%%%%%%%%%%%%%%%%%%%%

\begin{frame}[<+->]\frametitle{}

\textbf{The Gauss--Weierstrass kernel}
\begin{equation*}%\label{eq}
W(x)=c_d\exp\brac{-\frac{|x|^2}{4}},\quad c_d=(4\pi)^{-\frac{d}{2}},
\end{equation*}

\pause

The dilatations
\begin{equation*}%\label{eqRepresentationOfGWKernel}
W_t(x)=t^{-d}W\brac{\frac{x}{t}}=c_dt^{-d}\exp\brac{-\frac{|x|^2}{4t^2}}
\end{equation*}
are connected with heat equation.

\end{frame}

%%%%%%%%%%%%%%%%%%%%%%%%%%%%%%%%%%%%%%%%%%%%%%%%%%%%%%%%%%%%%%%%%%%
%%%%%%%%%%%%%%%%%%%%%%%%%%%%%%%%%%%%%%%%%%%%%%%%%%%%%%%%%%%%%%%%%%%
%%%%%%%%%%%%%%%%%%%%%%%%%%%%%%%%%%%%%%%%%%%%%%%%%%%%%%%%%%%%%%%%%%%

\section{Metric spaces}

%%%%%%%%%%%%%%%%%%%%%%%%%%%%%%%%%%%%%%%%%%%%%%%%%%%%%%%%%%%%%%%%%%%
%%%%%%%%%%%%%%%%%%%%%%%%%%%%%%%%%%%%%%%%%%%%%%%%%%%%%%%%%%%%%%%%%%%
%%%%%%%%%%%%%%%%%%%%%%%%%%%%%%%%%%%%%%%%%%%%%%%%%%%%%%%%%%%%%%%%%%%

\begin{frame}[<+->]\frametitle{}

When studying similar questions on metric spaces, it is not very right, perhaps, to consider the special shape of the kernels. It is more natural to clarify the conditions under which one or another type of convergence takes place.

\pause

More precisely, we consider the following questions. What are the conditions for

1) the family of kernels functions as an approximation of identity, 

2)what is the shape of the domains over which the function has a limit almost everywhere (Fatou property)?

\end{frame}

%%%%%%%%%%%%%%%%%%%%%%%%%%%%%%%%%%%%%%%%%%%%%%%%%%%%%%%%%%%%%%%%%%%

\begin{frame}[<+->]\frametitle{}

Let $X$ be � metric space with the metric $d$ and Borel measrure $\mu$, 

\pause

\begin{equation*}
B(x,r)=\{y\in X:d(x,y)<r\},
\end{equation*}
is open ball with center at the point $x\in X$ and the radius $r>0$, and assume always that measure of any ball $B\subset X$ is positive and finite. 

\end{frame}

%%%%%%%%%%%%%%%%%%%%%%%%%%%%%%%%%%%%%%%%%%%%%%%%%%%%%%%%%%%%%%%%%%%

\begin{frame}[<+->]\frametitle{}
	
We will assume that the doubling condition is satisfied, that is, there is such a number $a_{\mu}>0$, that
\begin{equation}\label{eqDoublingCondition}
\mu(B(x,2r))\le a_{\mu}\mu(B(x,r)),\quad  x\in X,\quad  r>0.
\end{equation}

\pause

It will provide us with standard $L^p$-inequalities for Hardy--Littlewood maximal functions. 

The regularity condition of the measure $\mu$ ensures the density of the class of continuous functions in spaces  $L^p(X)$, $1\le p<\infty$.	
	
\end{frame}


%%%%%%%%%%%%%%%%%%%%%%%%%%%%%%%%%%%%%%%%%%%%%%%%%%%%%%%%%%%%%%%%%%%
%%%%%%%%%%%%%%%%%%%%%%%%%%%%%%%%%%%%%%%%%%%%%%%%%%%%%%%%%%%%%%%%%%%
%%%%%%%%%%%%%%%%%%%%%%%%%%%%%%%%%%%%%%%%%%%%%%%%%%%%%%%%%%%%%%%%%%%

\section{Approximations of identity on metric measure spaces}

%%%%%%%%%%%%%%%%%%%%%%%%%%%%%%%%%%%%%%%%%%%%%%%%%%%%%%%%%%%%%%%%%%%
%%%%%%%%%%%%%%%%%%%%%%%%%%%%%%%%%%%%%%%%%%%%%%%%%%%%%%%%%%%%%%%%%%%
%%%%%%%%%%%%%%%%%%%%%%%%%%%%%%%%%%%%%%%%%%%%%%%%%%%%%%%%%%%%%%%%%%%

\begin{frame}[<+->]\frametitle{}

We consider families of integral operators
\begin{equation}\label{eqIntegralOperators}
\Phi_tf(x)=\int_{X}\varphi_t(x,z)f(z)\dmu(z),
\end{equation}
where the kernels $\varphi_t:X\times X\to\bbR$, $t>0$ forms approximation of identity, 

\pause

that is the following conditions are fulfilled: $\varphi_t\in L^{\infty}(X\times X)$,

\begin{equation}\label{eqIntegralOfPhiTIs1}
\int_{X}\varphi_t(x,z)\dmu(z)=1\quad\text{for all}\quad x\in X,\; t>0,
\end{equation}

\pause

\begin{equation}\label{eqIntegralsPhiTYAreBounded}
\sup_{t\in(0,1)}\sup_{x\in X}\int_X|\varphi_t(x,y)|\dmu(y)<\infty,
\end{equation}

\pause

\begin{equation}\label{eqIntegralsPhiTYOverCBTendsTo0}
\sup_{x\in X}\int_{d(x,y)>\delta}|\varphi_t(x,y)|\dmu(y)\to0\quad\text{if}\quad t\to+0\quad\delta>0,
\end{equation}

\end{frame}

%%%%%%%%%%%%%%%%%%%%%%%%%%%%%%%%%%%%%%%%%%%%%%%%%%%%%%%%%%%%%%%%%%%

\begin{frame}[<+->]\frametitle{}
	
The first question is solved simply and it is well known. 
	
The conditions \eqref{eqIntegralOfPhiTIs1}--\eqref{eqIntegralsPhiTYOverCBTendsTo0} are assumed to be fulfilled. Under these conditions, for any continuous function $f\in C(X)$
\begin{equation*}%\label{eq}
\Phi_tf(y)\to f(x)\quad\text{uniformly on } x\in X\;\text{ if } (y,t)\to (x,0).
\end{equation*}

\pause

That means $\Phi_tf$ is approximation of identity.	

\pause

We will focus on the Fatou property.	

\end{frame}

%%%%%%%%%%%%%%%%%%%%%%%%%%%%%%%%%%%%%%%%%%%%%%%%%%%%%%%%%%%%%%%%%%%
%%%%%%%%%%%%%%%%%%%%%%%%%%%%%%%%%%%%%%%%%%%%%%%%%%%%%%%%%%%%%%%%%%%
%%%%%%%%%%%%%%%%%%%%%%%%%%%%%%%%%%%%%%%%%%%%%%%%%%%%%%%%%%%%%%%%%%%

\section{$\lambda$-convergence and Fatou maximal operator}%\label{sc}

%%%%%%%%%%%%%%%%%%%%%%%%%%%%%%%%%%%%%%%%%%%%%%%%%%%%%%%%%%%%%%%%%%%
%%%%%%%%%%%%%%%%%%%%%%%%%%%%%%%%%%%%%%%%%%%%%%%%%%%%%%%%%%%%%%%%%%%
%%%%%%%%%%%%%%%%%%%%%%%%%%%%%%%%%%%%%%%%%%%%%%%%%%%%%%%%%%%%%%%%%%%

%%%%%%%%%%%%%%%%%%%%%%%%%%%%%%%%%%%%%%%%%%%%%%%%%%%%%%%%%%%%%%%%%%%

\begin{frame}[<+->]\frametitle{}

Let the function $\lambda:(0,1]\to(0,1]$, $\lambda(+0)=0$ be given. It defines the approach domains to the points on <<boundary>> of the abstract half space  $X\times(0,1]$
\begin{equation*}%\label{eq}
D_{\lambda}(x)=\{(y,t)\in X\times(0,1]:d(x,y)<\lambda(t)\}
\end{equation*}

\pause

Introduce the corresponding maximal operator
\begin{equation*}%\label{eq}
\caN_{\lambda}u(x):=\sup\{|u(y,t)|:(y,t)\in D_{\lambda}(x)\}
\end{equation*}

\end{frame}

%%%%%%%%%%%%%%%%%%%%%%%%%%%%%%%%%%%%%%%%%%%%%%%%%%%%%%%%%%%%%%%%%%%

\begin{frame}[<+->]\frametitle{}

We need the following additional condition on the measure $\mu$: there exists two constants $C_1,C_2>1$ such that 
\begin{equation}\label{eqComparatibilityOfBallMeasures}
\mu(B(x,C_1r))\ge C_2\mu(B(x,r)),\quad x\in X,\;r>0.
\end{equation}

\pause

This condition does not seem very strong.	
	
\end{frame}

%%%%%%%%%%%%%%%%%%%%%%%%%%%%%%%%%%%%%%%%%%%%%%%%%%%%%%%%%%%%%%%%%%%

\begin{frame}[<+->]%\frametitle{}

For further we need a new majorant
\begin{equation}\label{eqPhiStar}
\varphi_t^*(x,y):=\sup\{|\varphi_t(x,z)|:d(x,y)\le d(x,z)\},
\end{equation}

\pause

\begin{figure}[ht]
	\begin{center}	
		\includegraphics{picture2.pdf}
	\end{center}
	%\caption{}}\label{fg}
\end{figure}

\end{frame}

%%%%%%%%%%%%%%%%%%%%%%%%%%%%%%%%%%%%%%%%%%%%%%%%%%%%%%%%%%%%%%%%%%%

\begin{frame}[<+->]\frametitle{}

To study convergence almost everywhere we will use the following condition
\begin{equation}\label{eqLiNormOfPhiStar}
C_{\varphi}:=\sup_{t\in(0,1)}\sup_{x\in X}\norm{\varphi_t^*(x,\cdot)}{L^{1}(X)}<\infty,
\end{equation}	
which replaces the corresponding condition for a majorant in $\bbR^d$	
\end{frame}

%%%%%%%%%%%%%%%%%%%%%%%%%%%%%%%%%%%%%%%%%%%%%%%%%%%%%%%%%%%%%%%%%%%

\begin{frame}[<+->]\frametitle{}
	
\begin{theorem}\label{thEstimateOfFatouMaximalOperator}
	Let the conditions \eqref{eqDoublingCondition} and \eqref{eqComparatibilityOfBallMeasures} are fulfilled. If the function $\lambda$ satisfy the following condition
	\begin{equation}\label{eqMainCondition}
	C_{\lambda}:=\sup_{t\in(0,1)}\sup_{x\in X}\norm{\varphi_t^*(x,\cdot)}{L^{\infty}(X)}\mu(B(x,\lambda(t)))<\infty,
	\end{equation}
	then

	\begin{equation*}%\label{eq}
	\mu\brac{\{\caN_{\lambda}(\Phi_tf)>A\}}\les\dfrac{C_{\varphi}+C_{\lambda}}{A}\norm{f}{L^1(X)},\quad A>0,\;f\in L^1(X),
	\end{equation*}	
	and for any function $f\in L^1(X)$
	\begin{equation}\label{eqFatouConvergence}
	D_{\lambda}(x)-\lim \Phi_tf=f(x)\quad\text{for almost all}\quad x\in X.
	\end{equation}	
\end{theorem}
	
\end{frame}

%%%%%%%%%%%%%%%%%%%%%%%%%%%%%%%%%%%%%%%%%%%%%%%%%%%%%%%%%%%%%%%%%%%

\begin{frame}[<+->]\frametitle{}

We say that $x\in X$ is the Lebesgue point of the function  $f\in L^1(X)$, if
\begin{equation*}%\label{eq}
\lim_{t\to+0}\fint\limits_{B(x,t)}|f(y)-f(x)|\,d\mu(y)=0.
\end{equation*}	
	
\pause

\begin{theorem}\label{thConvergenceAtLebesguePoints}
	Under conditions \eqref{eqDoublingCondition}, \eqref{eqComparatibilityOfBallMeasures} and function $\lambda$ with property \eqref{eqMainCondition}, for any function $f\in L^1(X)$ we have
	\begin{equation*}%\label{eq}
	D_{\lambda}(x)-\lim \Phi_tf=f(x)
	\end{equation*}	
	at any Lebesgue point $x\in X$.
\end{theorem}
	
\end{frame}

%%%%%%%%%%%%%%%%%%%%%%%%%%%%%%%%%%%%%%%%%%%%%%%%%%%%%%%%%%%%%%%%%%%
%%%%%%%%%%%%%%%%%%%%%%%%%%%%%%%%%%%%%%%%%%%%%%%%%%%%%%%%%%%%%%%%%%%
%%%%%%%%%%%%%%%%%%%%%%%%%%%%%%%%%%%%%%%%%%%%%%%%%%%%%%%%%%%%%%%%%%%

\section{One-dimensional Poisson kernels in $\bbC$}

%%%%%%%%%%%%%%%%%%%%%%%%%%%%%%%%%%%%%%%%%%%%%%%%%%%%%%%%%%%%%%%%%%%
%%%%%%%%%%%%%%%%%%%%%%%%%%%%%%%%%%%%%%%%%%%%%%%%%%%%%%%%%%%%%%%%%%%
%%%%%%%%%%%%%%%%%%%%%%%%%%%%%%%%%%%%%%%%%%%%%%%%%%%%%%%%%%%%%%%%%%%

\begin{frame}[<+->]\frametitle{}
	
Let
\begin{equation*}%\label{}
p(z,\theta)=\frac{1}{2\pi}\cdot\frac{1-|z|^2}{|z-e^{i\theta}|^2}
\end{equation*}
be Poisson kernel in the unit circle $B$ of complex plane.	
	
\pause

A study of the convolutions with degrees of the Poisson kernel
\begin{equation}\label{eqPoissonIntegralWithExponent}
P_{l}f(z)=\int_{-\pi}^{\pi}\left[p(z,\theta)\right]^{l+\frac{1}{2}}f(\theta)\,d\theta,\;\;
l\ge0
\end{equation}
was initiated by Sj\"ogren, 1984.
	
\pause

It is interesting because
$P_{l}(z,\cdot)$ (and $P_{l}f(z)$) satisfy the following equation
\begin{equation*}%\label{}
\frac{1}{4}(1-|z|^2)^2\left(\frac{\partial^2u}{\partial
	x^2}+\frac{\partial^2u}{\partial y^2}\right)=
\left(l^2-\frac{1}{4}\right)u
\end{equation*}
	
\end{frame}

%%%%%%%%%%%%%%%%%%%%%%%%%%%%%%%%%%%%%%%%%%%%%%%%%%%%%%%%%%%%%%%%%%%

\begin{frame}[<+->]\frametitle{}

Let
\begin{equation}\label{eqNormedPoissonIntegralWithExponent}
\mathcal{P}_{l}f(z)=\frac{P_{l}f(z)}{P_{l}1(z)}
\end{equation}
Note that
\begin{equation}\label{eqAsymptoticOfPowerOfPoissonKernel}
P_{l}1(z)\asymp \left\{
\begin{array}{cr}
(1-|z|)^{\frac{1}{2}-l}, & l>0,\\
\\
(1-|z|)^{\frac{1}{2}}\log\frac{2}{1-|z|}, & l=0. \\
\end{array}
\right.
\end{equation}
	
\pause
 
For $l>0$ boundary behaviour of the integrals ${\mathcal P}_{l}f(z)$ is the same as for $l=\frac{1}{2}$:

\pause

Fatou domains are 
\begin{equation*}%\label{eq}
D(\varphi)=\left\{z\in\mathbb{C}:|z-e^{i\varphi}|<
a(1-|z|)\right\}.
\end{equation*}
	
\end{frame}

%%%%%%%%%%%%%%%%%%%%%%%%%%%%%%%%%%%%%%%%%%%%%%%%%%%%%%%%%%%%%%%%%%%

\begin{frame}[<+->]\frametitle{}
	
The case of $l=0$ is different: ${\mathcal P}_{l}f(z)$ converges to $f(e^{i\varphi})$, for every function $f\in L^1[-\pi,\pi]$ for almost all $\varphi\in[-\pi,\pi]$, if $z$ tends $e^{i\varphi}$ inside the domain
\begin{equation}\label{eqSjogrenDomains}
D(\varphi)=\left\{z\in\mathbb{C}:|z-e^{i\varphi}|<
a(1-|z|)\log\frac{2}{1-|z|}\right\}.
\end{equation}	
	
\pause
	
\end{frame}

%%%%%%%%%%%%%%%%%%%%%%%%%%%%%%%%%%%%%%%%%%%%%%%%%%%%%%%%%%%%%%%%%%%
%%%%%%%%%%%%%%%%%%%%%%%%%%%%%%%%%%%%%%%%%%%%%%%%%%%%%%%%%%%%%%%%%%%
%%%%%%%%%%%%%%%%%%%%%%%%%%%%%%%%%%%%%%%%%%%%%%%%%%%%%%%%%%%%%%%%%%%

\section{Multidimensional Poisson kernels in $\bbR^d$}

%%%%%%%%%%%%%%%%%%%%%%%%%%%%%%%%%%%%%%%%%%%%%%%%%%%%%%%%%%%%%%%%%%%
%%%%%%%%%%%%%%%%%%%%%%%%%%%%%%%%%%%%%%%%%%%%%%%%%%%%%%%%%%%%%%%%%%%
%%%%%%%%%%%%%%%%%%%%%%%%%%%%%%%%%%%%%%%%%%%%%%%%%%%%%%%%%%%%%%%%%%%

\begin{frame}[<+->]\frametitle{}

Let $X=S^{n-1}$ be a unit sphere in $\mathbb{R}^n$, $n\ge 2$,
$\mu$ be the surface  Lebesgue measure on $S^{n-1}$ normalized
by $\mu(S^{2n-1})=1$, $d(x,y)=|x-y|$ be the Euclidean metric. 

\pause

A multidimensional analogue of (\ref{eqPoissonIntegralWithExponent}) be the operator
\begin{equation*}%\label{}
P_{l}f(x)=\int_{S^{n-1}}\left[p(x,\eta)\right]^{l+\frac{n-1}{n}}f(\eta)\,d\mu(\eta),
\end{equation*}
where
\begin{equation*}%\label{}
p(x,\eta)=\frac{1-|x|^2}{\left|x-\eta\right|^{n}}
\end{equation*}
is the Poisson kernel for the unit ball and

\begin{equation*}%\label{eq}
\caP_lf(x)=\dfrac{P_{l}f(x)}{P_{l}1(x)}
\end{equation*}
	
\end{frame}

%%%%%%%%%%%%%%%%%%%%%%%%%%%%%%%%%%%%%%%%%%%%%%%%%%%%%%%%%%%%%%%%%%%

\begin{frame}[<+->]\frametitle{}

\begin{equation*}%\label{eqP0InRn}
\mathcal{P}_0f(x)\asymp\left(\log\dfrac{2}{1-|x|}\right)^{-1}
\int\limits_{S^{n-1}}\frac{f(\eta)}{|x-\eta|^{n-1}}\,d\mu(\eta),
\end{equation*}

\pause

\begin{equation*}%\label{eq}
D(\theta)=\fbrac{x=r\eta\in\bbR^n:|\theta-\eta|<a(1-r)\left(\log\frac{2}{1-r}\right)^{1/(n-1)}},\quad |\theta|=1
\end{equation*}
	
\end{frame}

%%%%%%%%%%%%%%%%%%%%%%%%%%%%%%%%%%%%%%%%%%%%%%%%%%%%%%%%%%%%%%%%%%%
%%%%%%%%%%%%%%%%%%%%%%%%%%%%%%%%%%%%%%%%%%%%%%%%%%%%%%%%%%%%%%%%%%%
%%%%%%%%%%%%%%%%%%%%%%%%%%%%%%%%%%%%%%%%%%%%%%%%%%%%%%%%%%%%%%%%%%%

\section{Multidimensional Poisson kernels in $\bbC^d$}

%%%%%%%%%%%%%%%%%%%%%%%%%%%%%%%%%%%%%%%%%%%%%%%%%%%%%%%%%%%%%%%%%%%
%%%%%%%%%%%%%%%%%%%%%%%%%%%%%%%%%%%%%%%%%%%%%%%%%%%%%%%%%%%%%%%%%%
%%%%%%%%%%%%%%%%%%%%%%%%%%%%%%%%%%%%%%%%%%%%%%%%%%%%%%%%%%%%%%%%%%%

\begin{frame}[<+->]\frametitle{}
	
Let $X=S^{2n-1}$ be a unit sphere in $\mathbb{C}^n=\mathbb{R}^{2n}$, $\mu$ be the Lebesgue surface measure, $\mu(S^{n-1})=1$. Let
$d(\zeta,\xi)=\left|1-\langle\zeta,\xi\rangle\right|$ be a nonisotropic
quasimetric (here $\langle\cdot,\cdot\rangle$
is the complex scalar product). 

\pause

Now it is natural to consider also
the invariant Poisson kernel
\begin{equation*}%\label{}
p(z,\zeta)=\frac{(1-|z|^2)^n}{\left|1-\langle z,\zeta\rangle\right|^{2n}}
\end{equation*}

\pause

\begin{equation*}%\label{}
P_{l}f(z)=\int_{S^{2n-1}}\left[p(z,\eta)\right]^{l+\frac{1}{2}}f(\eta)\,d\mu(\eta).
\end{equation*}

\pause

\begin{equation*}%\label{eq}
\caP_lf(x)=\dfrac{P_{l}f(z)}{P_{l}1(z)}
\end{equation*}
	
\end{frame}

%%%%%%%%%%%%%%%%%%%%%%%%%%%%%%%%%%%%%%%%%%%%%%%%%%%%%%%%%%%%%%%%%%%

\begin{frame}[<+->]\frametitle{}
	

\begin{equation*}%\label{eqP0InCn}
\mathcal{P}_0f(z)\asymp\left(\log\dfrac{2}{1-|z|}\right)^{-1}
\int\limits_{S^{2n-1}}\frac{f(\eta)}{|1-\langle z,\eta\rangle|^{n}}\,d\mu(\eta)
\end{equation*}

\pause

\begin{equation*}%\label{eq}
D(\zeta)=\fbrac{z=r\eta\in\bbC^n:|1-\langle z,\zeta\rangle|<a(1-r)\left(\log\frac{2}{1-r}\right)^{1/n}},\quad |\zeta|=1.
\end{equation*}
	
\end{frame}


%%%%%%%%%%%%%%%%%%%%%%%%%%%%%%%%%%%%%%%%%%%%%%%%%%%%%%%%%%%%%%%%%%%
%%%%%%%%%%%%%%%%%%%%%%%%%%%%%%%%%%%%%%%%%%%%%%%%%%%%%%%%%%%%%%%%%%%
%%%%%%%%%%%%%%%%%%%%%%%%%%%%%%%%%%%%%%%%%%%%%%%%%%%%%%%%%%%%%%%%%%%

\section{Poisson integrals for standard weighted Laplacians}

%%%%%%%%%%%%%%%%%%%%%%%%%%%%%%%%%%%%%%%%%%%%%%%%%%%%%%%%%%%%%%%%%%%
%%%%%%%%%%%%%%%%%%%%%%%%%%%%%%%%%%%%%%%%%%%%%%%%%%%%%%%%%%%%%%%%%%%
%%%%%%%%%%%%%%%%%%%%%%%%%%%%%%%%%%%%%%%%%%%%%%%%%%%%%%%%%%%%%%%%%%%

%%%%%%%%%%%%%%%%%%%%%%%%%%%%%%%%%%%%%%%%%%%%%%%%%%%%%%%%%%%%%%%%%%%

\begin{frame}[<+->]\frametitle{}
	
Let $\alpha>-1$

\begin{equation*}%\label{eq}
P_{\alpha}(z)=\dfrac{1}{2\pi}\brac{\dfrac{1-|z|^2}{1-\no{z}}}^{\alpha}\cdot\dfrac{1-|z|^2}{|1-z|^2}
\end{equation*}
\begin{equation*}%\label{eq}
\caP_{\alpha}f(z)=\int_{\bbT}P_{\alpha}(ze^{-i\tau})f(e^{i\tau})\,d\tau
\end{equation*}

\pause


the following equation corresponds to this operator 

\begin{equation*}%\label{eq}
\Delta_{\alpha}=\partial_z(1-|z|^2)^{-\alpha}\no{\partial}_z,
\end{equation*}

where 
\begin{equation*}%\label{eq}
\partial_z=\dfrac{1}{2}\brac{\pdr{}{x}+\dfrac{1}{i}\pdr{}{y}},\quad\no{\partial}_z=\dfrac{1}{2}\brac{\pdr{}{x}-\dfrac{1}{i}\pdr{}{y}},
\end{equation*}

\pause

The Fatou doamains are the same as for
usual Poisson kernel.
%
%\begin{equation*}%\label{eq}
%\left\{
%\begin{array}{lcl}
%\Delta_{\alpha}u=0, &  & |z|<1 \\
%\\
%u(e^{i\theta})=f(e^{i\theta}), &  & \theta\in\bbT
%\end{array}
%\right.
%\end{equation*}

	
\end{frame}

%%%%%%%%%%%%%%%%%%%%%%%%%%%%%%%%%%%%%%%%%%%%%%%%%%%%%%%%%%%%%%%%%%%

\begin{frame}[<+->]\frametitle{}
	
What about$L^p$-spaces with $p>1$?

\pause

For each operator above we know the right Fatou property.

\pause

But we not have general method. 
	
\end{frame}


%%%%%%%%%%%%%%%%%%%%%%%%%%%%%%%%%%%%%%%%%%%%%%%%%%%%%%%%%%%%%%%%%%%

\begin{frame}[<+->]\frametitle{}
	
	\begin{center}
		Thank you for attention!
	\end{center}
	
\end{frame}

%%%%%%%%%%%%%%%%%%%%%%%%%%%%%%%%%%%%%%%%%%%%%%%%%%%%%%%%%%%%%%%%%%%
%%%%%%%%%%%%%%%%%%%%%%%%%%%%%%%%%%%%%%%%%%%%%%%%%%%%%%%%%%%%%%%%%%%
%%%%%%%%%%%%%%%%%%%%%%%%%%%%%%%%%%%%%%%%%%%%%%%%%%%%%%%%%%%%%%%%%%%

\end{document}

%%%%%%%%%%%%%%%%%%%%%%%%%%%%%%%%%%%%%%%%%%%%%%%%%%%%%%%%%%%%%%%%%%%

\begin{frame}[<+->]\frametitle{}

\pause

\end{frame}

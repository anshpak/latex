\documentclass[a4paper,11pt]{article} %опция twoside нужна для разных колонтитулов; если на всех страницах нужен один колонтитул, убери

%%% Работа с русским языком
\usepackage{mathtext} 				% русские буквы в формулах
\usepackage[T2A]{fontenc}			% кодировка
\usepackage[utf8]{inputenc}			% кодировка исходного текста
\usepackage[english,russian]{babel}	% локализация и переносы

%%% Дополнительная работа с математикой
\usepackage{amsmath, amssymb, eucal}

\usepackage{geometry} % Простой способ задавать поля
\geometry{top=20mm}
\geometry{bottom=20mm}
\geometry{left=25mm}
\geometry{right=25mm}

%%% Кликабельные ссылки
\usepackage[pdftex, unicode, bookmarks, pagebackref]{hyperref}
% ДО этой строки можно копировать все смело!

%%% определение новых команд
\newcommand{\smb}[2]{\left(\frac{#1}{#2}\right)}
\newcommand{\xvec}[2][n]{ (#2_1,#2_2,\ldots,#2_#1) }

%%% Определение новых теорем
\newtheorem{predl}{Предложение}[subsection]
\newtheorem{predll}{Предложение}
%%% Выбор стиля страниц
\pagestyle{plain}

%%% Заголовок
\author{Создание новых команд}
\title{Лабораторная работа №5}
\date{06 мая 2022 г.}

% Посредством \renewcommand перекрываю печать значения счетчиков section и subsection арабскими цифрами (по умолчанию) римскими цифрами
\renewcommand{\thesection}{\Roman{section}} 
\renewcommand{\thesubsection}{\roman{subsection}}

% Новый счетчик zadacha, он не нумерует автоматически!
\newcounter{zadacha}
\setcounter{zadacha}{1}
\renewcommand{\thezadacha}{\Alph{zadacha}}

% Новая команда z
\newcommand{\z}[1]{\par \refstepcounter{zadacha} \textbf{Задача} \thezadacha \text{.} \\ \textit{#1} \label{\thezadacha}}

% Новая команда showMatrix
\newcommand{\showMatrix}[3]{\begin{pmatrix}
		#1_{11} & #1_{12} & \cdots & #1_{1 #2} \\
		#1_{21} & #1_{22} & \cdots & #1_{2 #2} \\
		\cdots & \cdots & \cdots & \cdots \\
		#1_{#3 1} & #1_{#3 2} & \cdots & #1_{#3 #2}
	\end{pmatrix}}

\begin{document} % Конец преамбулы, начало текста.
	
	\maketitle % печатает заголовок, имя автора и дату
	
	\section{Первая секция}
	
		\subsection{Первая подсекция первой секции}
		\subsection{Вторая подсекция первой секции}
		\subsection{Третья подсекция первой секции}
		
	\section{Вторая секция}
	
		\subsection{Первая подсекция второй секции}
		\subsection{Вторая подсекция второй секции}
		\subsection{Третья подсекция второй секции}
		
	\section{Третье задание}
	
		\thezadacha \\
		\stepcounter{zadacha}
		\thezadacha \\
		\stepcounter{zadacha}
		\thezadacha \\
		\stepcounter{zadacha}
		\thezadacha
	
		\z{Установите взаимное расположение плоскости П и плоскости П$_1$, заданной уравнениями: 
			\[
			\text{П}=\begin{cases}
				x_1 = t_1,\\
				x_2 = t_2,\\
				x_3 = 0,\\
				x_4 = 0,\\
				x_5 = 0,
			\end{cases}
			\]
			\[
			\text{П}_1=\begin{cases}
				x_1 = s_2 + 1,\\
				x_2 = s_2 - 1,\\
				x_3 = s_1 + s_2,\\
				x_4 = 1,\\
				x_5 = 0,
			\end{cases}
			\]
		}
	
		\z{Напишите алгоритм решения любой NP-задачи на выбор. Например:
			\begin{itemize}
				\item{По данному графу узнать, есть ли в нём клики (полные подграфы) заданного размера.}
				\item{Определить наличия в графе гамильтонова цикла.}
				\item{Cуществует ли маршрут не длиннее, чем заданное значение k.}
				\item{Узнать по данной булевой формуле, существует ли набор входящих в неё переменных, обращающий её в 1.}
			\end{itemize}
		}
	
		\z{Изобразить на листе бумаги формата А$4$ стандартный $5$-мерный симплекс $\delta_5$ вместе с барицентрической системой координат.} \\
		
		Та самая задача \ref{E} на 4 балла на экзамене по теории графов.
		
		Та самая задача \ref{F} на 4 балла на экзамене по геометрии.
		
		Задача \ref{G}.
		
		\section{Четвертое задание}
		
		$$\showMatrix{a}{n}{m}$$
		
		$$\showMatrix{x}{n}{m}$$
		
		$$\showMatrix{y}{k}{l}$$
		
\end{document} % Конец текста.
\documentclass[a4paper,12pt]{article} % добавить leqno в [] для нумерации слева

%%% Работа с русским языком
\usepackage{cmap}					% поиск в PDF
\usepackage{mathtext} 				% русские буквы в формулах
\usepackage[T2A]{fontenc}			% кодировка
\usepackage[utf8]{inputenc}			% кодировка исходного текста
\usepackage[english,russian]{babel}	% локализация и переносы

%%% Дополнительная работа с математикой
\usepackage{amsmath,amsfonts,amssymb,amsthm,mathtools} % AMS
\usepackage{icomma} % "Умная" запятая: $0,2$ --- число, $0, 2$ --- перечисление

%% Номера формул
%\mathtoolsset{showonlyrefs=true} % Показывать номера только у тех формул, на которые есть \eqref{} в тексте.

%%% Работа с таблицами
\usepackage{array,tabularx,tabulary,booktabs} % Дополнительная работа с таблицами
\usepackage{longtable}  % Длинные таблицы
\usepackage{multirow} % Слияние строк в таблице
\usepackage{wrapfig} % Обтекание рисунков и таблиц текстом

%% Шрифты
\usepackage{euscript}	 % Шрифт Евклид
\usepackage{mathrsfs} % Красивый матшрифт

%% Свои команды
\DeclareMathOperator{\sgn}{\mathop{sgn}}

%%% Заголовок
\author{Ссылки, таблицы и формулы в несколько строк}
\title{Лабораторная работа №2}
\date{\today}


\begin{document} % Конец преамбулы, начало текста.
	
	\maketitle % печатает заголовок, имя автора и дату
	
	
	
	
	
	\section{Первое задание}
	
	\section{Второе задание. Малые таблицы. 1}
	\begin{center}
	\begin{tabulary}{\textwidth}{|C|C|C|C|}
		\hline
		\multirow{2}{*}{Число делящихся элементов} & \multicolumn{3}{|c|}{Число частей при делении} \\ 
		\cline{2-4}
		& пространства плоскостями & плоскости прямыми & прямой точками \\ 
		\hline
		0 & 1 & 1 & 1 \\
		1 & 2 & 2 & 2 \\
		2 & 4 & 4 & 3 \\
		3 & 8 & 7 & 4 \\
		4 & 15 &  & 5 \\
		\dots & \dots & \dots & \dots \\
		$n$ &  &  & $n + 1$ \\
		\hline
	\end{tabulary}
	\end{center}
	
	\section{Третье задание}

%%%%%%%%%%%%%%%%%%%%%%%%%%%%%%%%%%%%%%%%%%%%%%%%%%%%%%%%%%%%%%%%%%%
	\section{Четвертое задание. Нумерация и системы. 1}
	5. Семи неравенствам
	\begin{equation}
		2x_1 + 3x_2 \leqslant 6,
	\end{equation}

	\begin{equation}
		x_1 + x_2 \leqslant 2, 
	\end{equation}

	\begin{equation}
		-x_1 - 3x_2 \leqslant 3,		
	\end{equation}

	\begin{equation}
		2x_1 \leqslant 3,		
	\end{equation}

	\begin{equation}
		-x_1 \leqslant 3,
	\end{equation}

	\begin{equation}
		-3x_1 + 7x_2 \leqslant 21,		
	\end{equation}

	\begin{equation}
		x_1 - 3x_2 \leqslant 3		
	\end{equation}
%%%%%%%%%%%%%%%%%%%%%%%%%%%%%%%%%%%%%%%%%%%%%%%%%%%%%%%%%%%%%%%%%%%
	\section{Пятое задание. Нумерация и системы. 2}
	75. 1) Пусть $f$ ~--- непрерывная на $X$ функция, $a, b \in R, a < b.$ Доказать, что функция
	\[
		f(a;b;x)=\begin{cases}
			f \left(x \right), &\text{если } a \leqslant f \left(x \right) \leqslant b, \\
		a, &\text{если } f \left(x \right) < a, \\
		b, &\text{если } f \left(x \right) > b,
		\end{cases}
	\]
	также непрерывна на $X.$
%%%%%%%%%%%%%%%%%%%%%%%%%%%%%%%%%%%%%%%%%%%%%%%%%%%%%%%%%%%%%%%%%%%
	\section{Шестое задание. Стандартные длинные формулы}
	С другой стороны известно, что монотонно возрастающая ограниченная последовательность чисел имеет конечный предел. Следовательно, если мы докажем, что последовательность чисел $x_n$ ограничена, то будет доказана и содимость ряда $\left( 26 \right)$. Положим
	\begin{multline*}
		y_{2n} = 1 - 
		\frac{1}{2^ \alpha} + 
		\frac{1}{3^ \alpha} -
		\frac{1}{4^ \alpha} +
		\frac{1}{5^ \alpha} -
		\frac{1}{6^ \alpha} + 
		\dots \\
		\dots 
		\frac{1}{ \left(2n - 1 \right)^ \alpha} -
		\frac{1}{ \left(2n \right)^ \alpha}.
	\end{multline*}
	Так как
	\begin{multline*}
		y_{2n} = 1 - 
		\left( 
			\frac{1}{2^ \alpha} - 
			\frac{1}{3^ \alpha}
		\right) -
		\left(
			\frac{1}{4^ \alpha} -
			\frac{1}{5^ \alpha} 
		\right) - 
		\dots - \\
		- \left(
			\frac{1}{ \left(2n - 2 \right)^ \alpha} -
			\frac{1}{ \left(2n - 1 \right)^ \alpha} 
		\right) -
		\frac{1}{ \left(2n \right)^ \alpha}.
	\end{multline*}
	то (числа в каждой скобке положительны)
	$$y_{2n} < 1.$$
	С другой стороны,
	\begin{multline*}
		y_{2n} = 1 - 
		\frac{1}{2^ \alpha} + 
		\frac{1}{3^ \alpha} -
		\frac{1}{4^ \alpha} +
		\frac{1}{5^ \alpha} -
		\frac{1}{6^ \alpha} + 
		\dots + \\
		+ \frac{1}{ \left(2n - 1 \right)^ \alpha} -
		\frac{1}{ \left(2n \right)^ \alpha} =
		\left(
			1 + 
			\frac{1}{2^ \alpha} + 
			\frac{1}{3^ \alpha} +
			\frac{1}{4^ \alpha} +
			\frac{1}{5^ \alpha} +
		\right. \\
		\left.
			+ \frac{1}{6^ \alpha} +
			\dots +
			\frac{1}{ \left(2n - 1 \right)^ \alpha} +
			\frac{1}{ \left(2n \right)^ \alpha}
		\right) - \\
		- 2
		\left(
			\frac{1}{2^ \alpha} + 
			\frac{1}{4^ \alpha} +
			\frac{1}{6^ \alpha} +
			\dots +
			\frac{1}{ \left(2n \right)^ \alpha}
		\right) = \\
		= \left(
			1 + 
			\frac{1}{2^ \alpha} + 
			\frac{1}{3^ \alpha} +
			\frac{1}{4^ \alpha} +
			\frac{1}{5^ \alpha} +
			\frac{1}{6^ \alpha} +
			\dots + 
		\right. \\
		\left.
			+ \frac{1}{ \left(2n - 1 \right)^ \alpha} +
			\frac{1}{ \left(2n \right)^ \alpha}
		\right)
		- \frac{2}{2^ \alpha}
		\left(
		1 +
		\frac{1}{2^ \alpha} + 
		\frac{1}{3^ \alpha} +
		\dots +
		\frac{1}{n^ \alpha}
		\right).
	\end{multline*}
	Так как 
	$x_n = 1 + 
	\dfrac{1}{2^ \alpha} + 
	\dfrac{1}{3^ \alpha} +
	\dots +
	\dfrac{1}{n^ \alpha},$
	то 
	$$y_{2n} = x_{2n} - \dfrac{2}{2^ \alpha} x_n.$$
%%%%%%%%%%%%%%%%%%%%%%%%%%%%%%%%%%%%%%%%%%%%%%%%%%%%%%%%%%%%%%%%%%%
	\section{Вопросы.}
	1. Как отформатировать date, чтобы в заголовке была нужная мне дата?
	
	2. Задание 4:
		1) Нужно ли добавлять номер страницы, как на скриншоте справа снизу?
		
		2) Нужно ли делать отступ с фразой "5. Семи неравенствам", как на скриншоте. И если нужно, то как?
		
		3) У меня получились большие отступы между неравенствами. Большой ли это недочет и как его исправить?
		
	3. Как сделать так, чтобы этот текст не слипался в абзацы? Как сделать так, чтобы самое первое предложение начиналось с абзаца? Это касается и заданий 6, 5
	4. Задание 6:
		1) Нет 100-процентного совпадения с оригиналом.
	
\end{document} % Конец текста.
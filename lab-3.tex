\documentclass[a4paper,12pt]{article} % добавить leqno в [] для нумерации слева

%%% Работа с русским языком
\usepackage{cmap}					% поиск в PDF
\usepackage{mathtext} 				% русские буквы в формулах
\usepackage[T2A]{fontenc}			% кодировка
\usepackage[utf8]{inputenc}			% кодировка исходного текста
\usepackage[english,russian]{babel}	% локализация и переносы

%%% Дополнительная работа с математикой
\usepackage{amsmath,amsfonts,amssymb,amsthm,mathtools} % AMS
\usepackage{icomma} % "Умная" запятая: $0,2$ --- число, $0, 2$ --- перечисление

%% Номера формул
%\mathtoolsset{showonlyrefs=true} % Показывать номера только у тех формул, на которые есть \eqref{} в тексте.

%%% Работа с таблицами
\usepackage{array,tabularx,tabulary,booktabs} % Дополнительная работа с таблицами
\usepackage{longtable}  % Длинные таблицы
\usepackage{multirow} % Слияние строк в таблице
\usepackage{wrapfig} % Обтекание рисунков и таблиц текстом

%% Шрифты
\usepackage{euscript}	 % Шрифт Евклид
\usepackage{mathrsfs} % Красивый матшрифт

%% Свои команды
\DeclareMathOperator{\sgn}{\mathop{sgn}}

%%% Заголовок
\author{Важные детали}
\title{Лабораторная работа №3}
\date{24 марта 2022 г.}


\begin{document} % Конец преамбулы, начало текста.
	
	\maketitle % печатает заголовок, имя автора и дату
	
	%%%%%%%%%%%%%%%%%%%%%%%%%%%%%%%%%%%%%%%%%%%%%%%%%%%%%%%%%%%%%%%
	\section{Первое задание.}	
	Пример 18. Вычислить интеграл
	\begin{equation}
		\displaystyle \int \limits_{-1}^{2} \dfrac{1 + x^2}{1 + x^4} \; dx. \tag{11}
	\end{equation}
	
	$\blacktriangle$
	Сделав при 
	$x \neq 0$
	замену переменного 
	$t = x - 1 / x$
	в соответствующем неопределенном интеграле, получим
	$$
		\displaystyle \int \dfrac{1 + x^2}{1 + x^4} \: dx =
		\int \dfrac{d \left(x - 1 / x \right)}{2 + \left(x - 1 / x \right)^2} = 
		\dfrac{1}{\sqrt{2}} \: \arctg \: \dfrac{x^2 - 1}{x \sqrt{2}} + C
	$$
	и, следовательно,
	%%%%%%%%%%%%%%%%%%%%%%%%%%%%%%%%%%%%%%%%%%%%%%%%%%%%%%%%%%%%%%%
	
	\newpage
	\section{Второе задание. Малые таблицы. 1}
	
	%%%%%%%%%%%%%%%%%%%%%%%%%%%%%%%%%%%%%%%%%%%%%%%%%%%%%%%%%%%%%%%
	
	\newpage
	\section{Третье задание}

	%%%%%%%%%%%%%%%%%%%%%%%%%%%%%%%%%%%%%%%%%%%%%%%%%%%%%%%%%%%%%%%
	
	\newpage
	\section{Четвертое задание. Нумерация и системы. 1}
	
	%%%%%%%%%%%%%%%%%%%%%%%%%%%%%%%%%%%%%%%%%%%%%%%%%%%%%%%%%%%%%%%
	
	\newpage
	\section{Пятое задание. Нумерация и системы. 2}
	
	%%%%%%%%%%%%%%%%%%%%%%%%%%%%%%%%%%%%%%%%%%%%%%%%%%%%%%%%%%%%%%%
	
	\newpage
	\section{Вопросы.}

\end{document} % Конец текста.